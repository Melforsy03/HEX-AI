
\documentclass[12pt]{article}
\usepackage[utf8]{inputenc}
\usepackage[spanish]{babel}
\usepackage{amsmath, amssymb}
\usepackage{graphicx}
\usepackage{hyperref}
\usepackage{geometry}
\usepackage{listings}
\usepackage{xcolor}
\geometry{a4paper, margin=2.5cm}
\title{Jugador Virtual Hex}
\author{Melani Forsythe Matos \\ Carrera: Ciencias de la Computación \\ Grupo: C312}
\date{}


\lstset{
  backgroundcolor=\color{white},
  basicstyle=\ttfamily\footnotesize,
  keywordstyle=\color{blue}\bfseries,
  commentstyle=\color{gray},
  stringstyle=\color{orange},
  showstringspaces=false,
  numbers=left,
  numberstyle=\tiny\color{gray},
  breaklines=true,
  frame=single,
  captionpos=b
}

\begin{document}

\maketitle

\section{Objetivo}
Desarrollar un jugador virtual competitivo para el juego de Hex capaz de enfrentar oponentes avanzados, tomando decisiones estratégicas de forma eficiente tanto en tableros pequeños como grandes.

\section{Estrategia General}
El jugador utiliza una combinación de:
\begin{itemize}
  \item Algoritmo \textbf{Minimax} con poda \textbf{alfa-beta} para tomar decisiones.
  \item \textbf{Evaluación heurística} adaptativa.
  \item \textbf{Reconocimiento de patrones estratégicos}: puentes, doble amenaza.
  \item \textbf{Paralelización} del cálculo usando múltiples núcleos.
  \item \textbf{Caching} de estados con tabla de transposición.
\end{itemize}

\section{Implementación Paso a Paso}

\subsection{1. Método principal: \texttt{play()}}
Este método decide la jugada a realizar:

\begin{lstlisting}[language=Python]
def play(self, tablero):
    movimientos = tablero.get_possible_moves()
    profundidad = self.calcular_profundidad(len(movimientos))
    args = [(self.jugador_id, tablero_flat, tam, mov, profundidad) for mov in movimientos]
    resultados = executor.map(evaluar_movimiento_global, args)
    return max(resultados)[1]
\end{lstlisting}

El método evalúa todas las jugadas posibles en paralelo y selecciona la mejor.

\subsection{2. Algoritmo \texttt{minimax()} con poda alfa-beta}

\begin{lstlisting}[language=Python]
def minimax(self, tablero, profundidad, maximizador, alfa, beta):
    if profundidad == 0 or tablero.check_connection(...):
        return self.evaluar(tablero)
    if maximizador:
        ...
    else:
        ...
\end{lstlisting}

Busca la mejor jugada minimizando el error y evitando explorar ramas inútiles.

\subsection{3. Evaluación heurística}

La función \texttt{evaluar()} es fundamental para el rendimiento e inteligencia del jugador virtual. Esta función analiza el estado actual del tablero y devuelve un valor numérico que indica qué tan buena es esa posición para el jugador actual. Cuanto mayor es el valor, más favorable es.

A continuación se describen cada uno de los componentes que conforman la heurística:

\begin{itemize}
  \item \textbf{Distancia mínima a la victoria}: Se calcula usando una versión simplificada de Dijkstra. Se trata de determinar la distancia más corta entre los dos bordes que el jugador necesita conectar. Esta distancia se penaliza o bonifica dependiendo de si es el jugador o su oponente.
  \begin{lstlisting}[language=Python]
  def distancia_conexion(jugador_id):
      dist = [[inf] * tam for _ in range(tam)]
      return min_distancia
  \end{lstlisting}

  \item \textbf{Amenaza del oponente}: Si el oponente está a una sola jugada de ganar, se devuelve un valor muy negativo. Esto fuerza al jugador a bloquear esa amenaza en lugar de hacer otra jugada.
  \begin{lstlisting}[language=Python]
  if distancia_oponente <= 1:
      return -1000
  \end{lstlisting}

  \item \textbf{Control del centro}: Las casillas cerca del centro del tablero tienen más conexiones posibles, por lo tanto, se bonifican.
  \begin{lstlisting}[language=Python]
  for fila in range(tam):
      for col in range(tam):
          if tablero[fila][col] == jugador:
              score += max(0, 5 - distancia_al_centro)
  \end{lstlisting}

  \item \textbf{Puentes seguros}: El jugador recibe puntos extra si hay dos fichas propias conectadas indirectamente con un hueco en el medio (estructura típica de Hex).
  \begin{lstlisting}[language=Python]
  if tablero[nf][nc] == jugador and tablero[mf][mc] == 0:
      bonus += 1
  \end{lstlisting}

  \item \textbf{Amenazas dobles}: Si una jugada conecta dos grupos propios separados, se cuenta como una doble amenaza, lo cual complica la defensa del oponente.
  \begin{lstlisting}[language=Python]
  if conexiones_vecinas >= 2:
      doble_amenaza += 1
  \end{lstlisting}

  \item \textbf{Adaptación a la etapa de juego}:
  \begin{itemize}
    \item En la \textbf{apertura}, se favorece el control del centro y el desarrollo.
    \item En el \textbf{medio juego}, se balancea entre atacar y defender.
    \item En el \textbf{final}, se prioriza la conexión directa y bloquear.
  \end{itemize}
\end{itemize}


\textbf{Función completa de evaluación:}
\begin{lstlisting}[language=Python]
return (d_oponente - d_propio) * peso + 
       centro_score +
       puente_bonus * 3 +
       doble_amenaza * 5 +
       amenaza_oponente
\end{lstlisting}

Donde:
\begin{itemize}
  \item $d_{propio}$ y $d_{oponente}$ son distancias mínimas a la conexión (Dijkstra).
  \item \texttt{puente\_bonus}: detecta estructuras tipo \texttt{X \_ X}.
  \item \texttt{doble\_amenaza}: jugadas que conectan múltiples grupos.
  \item \texttt{amenaza\_oponente}: penaliza si el oponente está a una jugada de ganar.
\end{itemize}

\subsection{4. Caching con tabla de transposición}

\begin{lstlisting}[language=Python]
key = hash(str(tablero.tablero))
if key in transposition_table:
    return transposition_table[key]
\end{lstlisting}

Se evita evaluar múltiples veces la misma posición del tablero.

\section{Características Avanzadas}
\begin{itemize}
  \item \textbf{Puentes seguros}: conexión indirecta que garantiza continuidad.
  \item \textbf{Amenazas dobles}: forzan al oponente a dividir su defensa.
  \item \textbf{Control del centro}: bonificación en las primeras jugadas.
  \item \textbf{Etapas del juego}: evaluación adaptativa (inicio, medio, final).
\end{itemize}

\section{Experiencia y Ajustes}
Durante las pruebas se observó:

\begin{itemize}
  \item Jugador 2 parecía tener ventaja debido a las respuestas inmediatas: esto se equilibró mejorando la apertura del jugador 1 (jugada central).
  \item Jugadas repetitivas entre jugadores: solucionado añadiendo evaluación contextual (dobles amenazas, puentes).
  \item Se identificó la necesidad de caching y se optimizó.
\end{itemize}

\end{document}